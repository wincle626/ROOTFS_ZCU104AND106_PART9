% ======================================================================
% scrlayer-notecolumn.tex
% Copyright (c) Markus Kohm, 2013-2019
%
% This file is part of the LaTeX2e KOMA-Script bundle.
%
% This work may be distributed and/or modified under the conditions of
% the LaTeX Project Public License, version 1.3c of the license.
% The latest version of this license is in
%   http://www.latex-project.org/lppl.txt
% and version 1.3c or later is part of all distributions of LaTeX 
% version 2005/12/01 or later and of this work.
%
% This work has the LPPL maintenance status "author-maintained".
%
% The Current Maintainer and author of this work is Markus Kohm.
%
% This work consists of all files listed in manifest.txt.
% ----------------------------------------------------------------------
% scrlayer-notecolumn.tex
% Copyright (c) Markus Kohm, 2013-2019
%
% Dieses Werk darf nach den Bedingungen der LaTeX Project Public Lizenz,
% Version 1.3c, verteilt und/oder veraendert werden.
% Die neuste Version dieser Lizenz ist
%   http://www.latex-project.org/lppl.txt
% und Version 1.3c ist Teil aller Verteilungen von LaTeX
% Version 2005/12/01 oder spaeter und dieses Werks.
%
% Dieses Werk hat den LPPL-Verwaltungs-Status "author-maintained"
% (allein durch den Autor verwaltet).
%
% Der Aktuelle Verwalter und Autor dieses Werkes ist Markus Kohm.
% 
% Dieses Werk besteht aus den in manifest.txt aufgefuehrten Dateien.
% ======================================================================
%
% Chapter about scrlayer-notecolumn of the KOMA-Script guide
% Maintained by Markus Kohm
%
% ----------------------------------------------------------------------
%
% Kapitel ueber scrlayer-notecolumn in der KOMA-Script-Anleitung
% Verwaltet von Markus Kohm
%
% ============================================================================

\KOMAProvidesFile{scrlayer-notecolumn.tex}
                 [$Date: 2018-02-05 01:50:56 -0800 (Mon, 05 Feb 2018) $
                  KOMA-Script guide (chapter:scrlayer-notecolumn)]

\translator{Markus Kohm\and Arndt Schubert\and Karl Hagen}

% Date of the translated German file: 2019-12-03

\chapter{Note Columns with \Package{scrlayer-notecolumn}}
\labelbase{scrlayer-notecolumn}

\BeginIndexGroup
\BeginIndex{Package}{scrlayer-notecolumn}%
Through version~3.11b, \KOMAScript{} supported note columns only in the form
of marginal notes that get their contents from \DescRef{maincls.cmd.marginpar}
and \DescRef{maincls.cmd.marginline} (see \autoref{sec:maincls.marginNotes},
\DescPageRef{maincls.cmd.marginline}). This kind of note column has several
disadvantages:
\begin{itemize}
\item Marginal notes must be set completely on a single page. Page
  breaks\textnote{page break} inside marginal notes are not possible. This
  sometimes causes the marginal notes to protrude into the lower margin.
\item Marginal notes near page breaks sometimes float to the next page and
  then, in the case of two-sided printing, cause alternate marginal columns to
  appear in the wrong margin.\textnote{assignment to the correct margin}. This
  problem can be solved with the additional package
  \Package{mparhack}\IndexPackage{mparhack} or by using \Macro{marginnote}
  from the \Package{marginnote}\IndexPackage{marginnote} package (see
    \cite{package:marginnote}).
\item Marginal notes inside floating environments\textnote{floating
    environments} or footnotes\textnote{footnotes} are not possible. This
  problem can also be solved with \Macro{marginnote} of the
  \Package{marginnote}\IndexPackage{marginnote} package.
\item There is only one marginal note column\textnote{several note columns},
  or at most two if you use \Macro{reversemarginpar} and
  \Macro{normalmarginpar}. Note that \Macro{reversemarginpar} is of less
  utility with two-sided documents.
\end{itemize}
Using \Package{marginnote}\IndexPackage{marginnote} leads to one more problem.
Because the package does not have any collision detection, marginal notes that
are set near to each other can partially or totally overlap. Moreover,
depending on the settings used, \Macro{marginnote} sometimes changes the
baseline distance of the normal text.

The \Package{scrlayer-notecolumn} package should solve all these problems. To
do so, it relies on the basic functionality of
\hyperref[cha:scrlayer]{\Package{scrlayer}}\IndexPackage{scrlayer}%
\important{\hyperref[cha:scrlayer]{\Package{scrlayer}}}. However, using this
package has a drawback:\textnote{Attention!} you can only output notes on
pages that use a page style based on
\hyperref[cha:scrlayer]{\Package{scrlayer}}. This disadvantage, however, can
easily be resolved, or even turned into an advantage, with the help of
\hyperref[cha:scrlayer-scrpage]{\Package{scrlayer-scrpage}}%
\important{\hyperref[cha:scrlayer-scrpage]{\Package{scrlayer-scrpage}}}%
\IndexPackage{scrlayer-scrpage}.

\section{Note about the State of Development}
\seclabel{draft}

This package was originally developed as a so-called \emph{proof of
  concept}\textnote{proof of concept} to demonstrate the potential of
\hyperref[cha:scrlayer]{\Package{scrlayer}}%
\important{\hyperref[cha:scrlayer]{\Package{scrlayer}}}. Although it is still
in its early stages of development, most of its stability is less a question
of \Package{scrlayer-notecolumn} than of
\hyperref[cha:scrlayer]{\Package{scrlayer}}. Nevertheless, you can assume that
there are still bugs in \Package{scrlayer-notecolumn}. Please report such bugs
whenever you find them. Some of the package's shortcomings are caused by the
attempt to minimise complexity. For example, although note columns can break
across several pages, there is no new paragraph break. \TeX{} itself does not
provide this.

Because the package is rather experimental\textnote{experimental}, its
instructions are found here in the second part of \iffree{the \KOMAScript{}
manual}{this book}. Accordingly, it is primarily directed towards experienced
users. If you are a beginner or a user on the way to become an
expert\textnote{for experts}, some of the following explanations may be
unclear or even incomprehensible.
\iffree{Please understand that I want to keep the effort spent on the manual
  to something halfway bearable when it comes to experimental packages.}{%
  However, this should be enough to solve simple tasks with
  \Package{scrlayer-notecolumn}. At the same time, developers will hopefully
  find useful stimulus for their own ideas.}

\iffalse% Umbruchoptimierung
Note also\textnote{Attention!} that the author of \KOMAScript{} does not
guarantee the further development of this package and offers only limited
support for it. This is the nature of a package written solely for the
purposes of demonstration.%
\fi

\LoadCommonFile{options}% \section{Early or late Selection of Options}

\LoadCommonFile{textmarkup}% \section{Text Markup}

\section{Declaring New Note Columns}
\seclabel{declaration}

Loading the package automatically declares a note column named
\PValue{marginpar}. As the name implies, this note column is placed in the
area of the normal marginal column used by \DescRef{maincls.cmd.marginpar} and
\DescRef{maincls.cmd.marginline}. The \Macro{reversemarginpar} and
\Macro{normalmarginpar} settings are also taken into account, but only for all
the notes on a page instead of note by note. The relevant setting is the one
that applies at the end of the page, namely during the output of the note
column. If you want to have notes in both the left and right margin of the
same page, you should define a second note column.

The default settings for all newly declared note columns are the same as the
defaults for \PValue{marginpar}. %
\iftrue% Umbruchoptimierung
But you can easily change them during their initialisation.%
\fi

Note\textnote{Attention!} that note columns can be output only on pages that
use a page style based on the
\hyperref[cha:scrlayer]{\Package{scrlayer}}\IndexPackage{scrlayer}%
\important{\hyperref[cha:scrlayer]{\Package{scrlayer}}} package. The
\Package{scrlayer-notecolumn} package automatically loads
\hyperref[cha:scrlayer]{\Package{scrlayer}}, which by default provides only
\PageStyle{empty}\IndexPagestyle{empty} page style. If you need additional
page styles, \hyperref[cha:scrlayer-scrpage]{\Package{scrlayer-scrpage}}%
\IndexPackage{scrlayer-scrpage}%
\important{\hyperref[cha:scrlayer-scrpage]{\Package{scrlayer-scrpage}}} is
recommended.

\begin{Declaration}
  \Macro{DeclareNoteColumn}%
  \OParameter{option~list}\Parameter{note~column~name}%
  \Macro{DeclareNewNoteColumn}%
  \OParameter{option~list}\Parameter{note~column~name}%
  \Macro{ProvideNoteColumn}%
  \OParameter{option~list}\Parameter{note~column~name}%
  \Macro{RedeclareNoteColumn}%
  \OParameter{option~list}\Parameter{note~column~name}%
\end{Declaration}
You can use these commands to create note columns. \Macro{DeclareNoteColumn}
creates the note column regardless of whether it already exists.
\Macro{DeclareNewNoteColumn} throws an error if the \PName{note column name}
has already been used for another note column. \Macro{ProvideNoteColumn}
simply does nothing in that case. You can use \Macro{RedeclareNoteColumn} only
to reconfigure an existing note column.

By the way, when reconfiguring existing note columns with
\Macro{DeclareNoteColumn} or \Macro{RedeclareNoteColumn}, the notes that have
already been generated for this column are retained.

\BeginIndex{FontElement}{notecolumn.\PName{note column name}}%
\BeginIndex{FontElement}{notecolumn.marginpar}%
Declaring a new note column always defines a new element for changing its font
attributes with \DescRef{\LabelBase.cmd.setkomafont} and
\DescRef{\LabelBase.cmd.addtokomafont}, if such an element does not yet exist.
The name of the element is \PValue{notecolumn.}\PName{note column name}. For
this reason, the default note column \PValue{marginnote} has the
element\textnote{element name} \FontElement{notecolumn.marginpar}. You can
directly specify the initial setting of the element's font when declaring a
note column by using the \Option{font} option within the \PName{option list}.%
\EndIndex{FontElement}{notecolumn.marginpar}%
\EndIndex{FontElement}{notecolumn.\PName{note column name}}%

The \PName{option list} is a comma-separated list of keys with or without
values, also known as options. The available options are listed in
\autoref{tab:scrlayer-notecolumn.note.column.options},
\autopageref{tab:scrlayer-notecolumn.note.column.options}.
The\textnote{default: option \Option{marginpar}} \Option{marginpar} option is
set by default, but you can overwrite this default with your individual
settings.

Because note columns are implemented using \Package{scrlayer}, a
layer\Index{layer} is created for each note column. The layer
name\textnote{layer name} is the same as the name of the element,
\PValue{notecolumn.}\PName{note column name}. For more information about
layers see \autoref{sec:scrlayer.layers}, starting on
\autopageref{sec:scrlayer.layers}.
%
\begin{Example}
  Suppose you are a professor of comedy law and want to write a treatise on
  the new ``Statute Concerning the Riotous Airing of Common Humour'', SCRACH
  for short. The better part of the work will consist of commentary on
  individual paragraphs of the statute. You decide on a two-column layout,
  with the comments in the main column and the paragraphs placed in a smaller
  note column on the right of the main column using a font that is
  smaller\iffree{ and in a different colour}{}.
\begin{lstcode}
  \documentclass{scrartcl}
  \usepackage{lmodern}
  \usepackage{xcolor}

  \usepackage{scrjura}
  \setkomafont{contract.Clause}{\bfseries}
  \setkeys{contract}{preskip=-\dp\strutbox}

  \usepackage{scrlayer-scrpage}
  \usepackage{scrlayer-notecolumn}

  \newlength{\paragraphscolwidth}
  \AfterCalculatingTypearea{%
    \setlength{\paragraphscolwidth}{%
      .333\textwidth}%
    \addtolength{\paragraphscolwidth}{%
      -\marginparsep}%
  }
  \recalctypearea
  \DeclareNewNoteColumn[%
    position=\oddsidemargin+1in
             +.667\textwidth
             +\marginparsep,
    width=\paragraphscolwidth,
    font=\raggedright\footnotesize
         \color{blue}
  ]{paragraphs}
\end{lstcode}
  The treatise should be a one-sided article. The font is Latin Modern, and
  the colour selection uses the \Package{xcolor}\IndexPackage{xcolor} package.

  For formatting legal texts\textnote{legal texts with
  \hyperref[cha:scrjura]{\Package{scrjura}}} with the
  \hyperref[cha:scrjura]{\Package{scrjura}}\IndexPackage{scrjura} package, see
  \autoref{cha:scrjura}.

  Since this document uses a page style\textnote{page style with
    \hyperref[cha:scrlayer-scrpage]{\Package{scrlayer-scrpage}}} with a
  page number, the
  \hyperref[cha:scrlayer-scrpage]{\Package{scrlayer-scrpage}}%
  \IndexPackage{scrlayer-scrpage} package is loaded. Thus, note columns can be
  output on all pages.

  Next, the \Package{scrlayer-notecolumn}\textnote{note columns with
  \Package{scrlayer-notecolumn}} package is loaded. The required width of the
  note column is calculated with
  \DescRef{typearea-experts.cmd.AfterCalculatingTypearea}%
  \IndexPackage{typearea}\IndexCmd{AfterCalculatingTypearea} after any
  recalculation\textnote{type area with
  \hyperref[cha:typearea]{\Package{typearea}}}%
  \IndexPackage{typearea} of the type area. It should be one third of the type
  area minus the distance between the main text and the note column. %

  With this information, we define the new note column. For the positions, we
  use a simple dimension expression. Note that \Length{oddsidemargin} is not
  the total left margin but for historical reasons the left margin minus
  1\Unit{inch}. So we have to add this value.

  This concludes the definition. Note that the note column would currently be
  placed inside the type area. This means that the note column would overwrite
  the text.

\begin{lstcode}
  \begin{document}

  \title{Commentary on the SCRACH}
  \author{Professor R. O. Tenase}
  \date{11/11/2011}
  \maketitle
  \tableofcontents

  \section{Preamble}
  The SCRACH is without doubt the most important law
  on the manners of humour that has been passed in
  the last thousand years. The first reading took
  place on 11/11/1111 in the Supreme Manic Fun
  Congress, but the law was rejected by the Vizier
  of Fun. Only after the ludicrous, Manic Fun
  monarchy was transformed into a representative,
  witty monarchy by W. Itzbold, on 9/9/1999 was
  the way finally clear for this law.
\end{lstcode}
  Because\textnote{Attention!} the text area was not reduced, the preamble is
  output extending over the whole width of the type area. To test this, you
  can temporarily add:
\begin{lstcode}
  \end{document}
\end{lstcode}
\end{Example}
%
In the example, the question of how the text for the commentary can be set in
a narrower column remains unresolved. You will discover how to do this by
continuing the example below.%
%
%\begin{table}% Umbruchoptimierung: Tabelle hinter das Beispiel verschoben
\begin{desclist}
  \desccaption{%
%  \caption[Available settings for declaring note columns]{%
    Available settings for declaring note columns%
%  }%
    \label{tab:scrlayer-notecolumn.note.column.options}%
  }{%
    Available settings for declaring note columns
    (\emph{continued})%
  }%
%  \begin{desctabular}
    \entry{\OptionVName{font}{font attribute}}{%
      The font attributes of the note column set with
      \DescRef{\LabelBase.cmd.setkomafont}. For possible values, refer to
      \autoref{sec:maincls.textmarkup},
      \DescPageRef{maincls.cmd.setkomafont}.\par%
      Default: \emph{empty}%
    }%
    \entry{\Option{marginpar}}{%
      Sets \Option{position} and \Option{width} to correspond to the marginal
      note column of \DescRef{maincls.cmd.marginpar}. Switching between
      \Macro{reversemarginpar}\IndexCmd{reversemarginpar} and
      \Macro{normalmarginpar}\IndexCmd{normalmarginpar} is only considered at
      the end of the page when the note column is output. Note that this
      option does not expect or allow any value.\par%
      Default: \emph{yes}%
    }%
    \entry{\Option{normalmarginpar}}{%
      Sets \Option{position} and \Option{width} to use the normal marginal
      note column and ignore \Macro{reversemarginpar} and
      \Macro{normalmarginpar}. Note that this option does not expect or allow
      a value.\par%
      Default: \emph{no}%
    }%
    \entry{\OptionVName{position}{offset}}{%
      Sets the horizontal offset of the note column from the left edge of the
      paper. The \PName{offset} can be a complex expression as long as it is
      fully expandable and expands to a length or a dimensional expression at
      the time the note column is output. See \cite[section~3.5]{manual:eTeX}
      for more information about dimensional expressions.\par%
      Default: \emph{through the \Option{marginpar} option}%
    }%
    \entry{\Option{reversemarginpar}}{%
      Sets \Option{position} and \Option{width} to use the reverse marginal
      note column of \DescRef{maincls.cmd.marginpar} with the
      \Macro{reversemarginpar} setting. Note that this option does not expect
      or allow a value.\par%
      Default: \emph{no}%
    }%
    \entry{\OptionVName{width}{length}}{%
      Sets the width of the note column. The \PName{length} can be a complex
      expression as long as it is fully expandable and expands to a length or
      a dimensional expression at the time the note column is output. See
      \cite[section~3.5]{manual:eTeX} for more information about dimensional
      expressions.\par%
      Default: \emph{through the \Option{marginpar} option}%
    }%
%  \end{desctabular}
%\end{table}
\end{desclist}
\EndIndexGroup


\section{Making a Note}
\seclabel{makenote}

After you declare a note column, you can create notes for this column. But
these notes are not be output immediately. Initially, they are written to an
auxiliary file with extension ``\File{.slnc}''. Specifically, they are first
written to the \File{aux}-file and, when the \File{aux}-file is read inside
\Macro{end}\PParameter{document}, they are copied to the \File{slnc}-file. If
necessary, the \Macro{nofiles}\IndexCmd{nofiles} setting is also taken into
account. At the next \LaTeX{} run, this auxiliary file will be read piece by
piece, according to the progress of the document, and at the end of the page
the notes for that page will be output.

Note\textnote{Attention!}, however, that note columns are output only on pages
whose page style is based on the \Package{scrlayer}\IndexPackage{scrlayer}
package. This package is loaded automatically by \Package{scrlayer-notecolumn}
and by default provides only the \PageStyle{empty}\IndexPagestyle{empty} page
style. If you need additional page styles, the
\Package{scrlayer-scrpage}\IndexPackage{scrlayer-scrpage} package is
recommended.

\begin{Declaration}
  \Macro{makenote}\OParameter{note-column name}\Parameter{note}\\
  \Macro{makenote*}\OParameter{note-column name}\Parameter{note}\\
\end{Declaration}
You can use this command to make a new \PName{note}. The current vertical
position is used as the vertical position for the start of the \PName{note}.
The horizontal position for the note results from the defined position of the
note column. To work correctly, the package relies on
\Macro{pdfsavepos}\IndexCmd{pdfsavepos},
\Macro{pdflastypos}\IndexCmd{pdflastypos}, and
\Macro{pdfpageheight}\IndexLength{pdfpageheight} or their equivalent in newer
Lua\TeX{} versions. Without these commands, \Package{scrlayer-notecolumn} will
not work. The primitives should act exactly as they would using pdf\TeX.

However, if the package detects a collision\textnote{collision avoidance} with
a previous \PName{note} in the same note column, the new \PName{note} is moved
below that earlier \PName{note}. If the \PName{note} does not fit on the
page\textnote{page break}\Index{page>break}, it will be moved completely or
partially to the next page.

The optional argument \PName{note column name} determines which note column
should be used for the \PName{note}. If the optional argument is omitted, the
default note column \PValue{marginpar} is used.%
\begin{Example}
  Let's add a commented paragraph to the example of the previous section. The
  paragraph itself should be placed in the newly defined note column:
\begin{lstcode}
  \section{Analysis}
  \begin{addmargin}[0pt]{.333\textwidth}
    \makenote[paragraphs]{%
      \protect\begin{contract}
        \protect\Clause{%
          title={No Joke without an Audience}%
        }
        A joke can only be funny if is has an
        audience.
      \protect\end{contract}%
    }
    This is one of the most central statements of
    the law. It is so fundamental that it is quite
    appropriate to bow to the wisdom of the authors.
\end{lstcode}
  The \DescRef{maincls.env.addmargin}\IndexEnv{addmargin}%
  \textnote{column width with \DescRef{maincls.env.addmargin}} environment,
  which is described in \autoref{sec:maincls.lists},
  \DescPageRef{maincls.env.addmargin}, is used to reduce the width of the main
  text by the width of the column for the paragraphs.

  Here you can see one of the few problems of using \Macro{makenote}. Because
  the mandatory argument is written to an auxiliary file,
  commands\textnote{fragile commands} inside this argument can, unfortunately,
  \emph{break}. To avoid this, you should use \Macro{protect} in front of all
  commands that should not expand when written to the auxiliary file.
  Otherwise, using a command inside this argument could result in error
  messages.

  In principle you could now finish this example with
\begin{lstcode}
  \end{addmargin}
  \end{document}
\end{lstcode}
  to see a preliminary result.
\end{Example}
If you test this example, you will see that the column for the legal text is
longer than that of the commentary. If\textnote{Attention!} you add another
section with another paragraph, you may encounter the problem that the
commentary will not continue below the legal text but immediately after the
previous comment. Next you will find a solution to this problem.

The\ChangedAt{v0.1.2583}{\Package{scrlayer-notecolumn}} problem with fragile
commands mentioned in the example above does not occur with the starred
variant\important{\Macro{makenote*}}. It uses \Macro{detokenize} to prevent
the expansion of commands. But this also means that you should not use
commands in the \PName{note} that change their definition within the
document.

Unfortunately, both commands have two other known
limitations\textnote{Attention!}. The first issue is related to colours using
\Package{color}\IndexPackage{color} or \Package{xcolor}\IndexPackage{xcolor}
within the note columns. To make such colour changes possible each note column
requires its own colour management using so-called \emph{colour stacks}.
Because the package was designed only as a \emph{proof of concept} and because
\XeTeX{} does not support multiple colour stacks, \XeTeX{} colour switching is
restricted to the attributes of the font element
\FontElement{notecolumn.\PName{note column name}}, a limitation which
eliminates the time and effort required to implement custom colour management.

The second issue is of a more conceptual nature. The content of the auxiliary
file that contains the note-column information is read while processing the
page header. This has consequences in particular if the read occurs while
an environment like \Environment{verbatim} is active. In this case, the
\Macro{catcode} settings of this environment would be active while reading the
auxiliary file. This will inevitably lead to errors in processing and output.
To attenuate this risk, the \Macro{catcodes} of the characters from
\Macro{dospecials}\IndexCmd{dospecials} are stored during
\Macro{begin}\PParameter{document} and explicitly restored when reading from
the auxiliary file.%
\EndIndexGroup


\begin{Declaration}
  \Macro{syncwithnotecolumn}\OParameter{note column name}
\end{Declaration}
This command adds a
synchronisation\textnote{synchronisation}\Index{synchronisation} point in a
note column and in the main text of the document. Whenever a synchronisation
point is reached during the output of a note column or the main text, a mark
will be generated that consists of the current page and the current vertical
position.

In parallel with the generation of synchronisation points,
\Package{scrlayer-notecolumn} determines whether a mark has been set in the
note column or the main text during the previous \LaTeX{} run. If so, it
compares their values. If the mark of the note column is lower on the current
page or on a later page, the main text will be moved down to the position of
the mark.

As a rule,\textnote{Hint!} you should not place synchronisation points within
paragraphs of the main text but only between them. If you nonetheless use
\Macro{syncwithnotecolumn} inside a paragraph, the synchronisation point will
be delayed until the current line has been output. This behaviour is similar
to that of, e.\,g., \Macro{vspace}\IndexCmd{vspace} in this respect.

Because synchronisation points are not recognized until the next \LaTeX{}
run\textnote{several \LaTeX{} runs}, this mechanism requires at least three
\LaTeX{} runs. Any new synchronisation may also result in shifts of later
synchronisation points, which in turn will require additional \LaTeX{} runs.
Such shifts are usually indicated by the message: ``\LaTeX{} Warning: Label(s)
may have changed. Rerun to get cross-references right.'' But reports about
undefined labels may also indicate the need for another \LaTeX{} run.

If you omit the optional argument, the default note column \PValue{marginpar}
will be used. Note\textnote{Attention!} that an empty optional argument is not
the same as omitting the optional argument!

You cannot use\textnote{Attention!} \Macro{syncwithnotecolumn} inside a note
itself, that is, inside the mandatory argument of
\Macro{makenote}\IndexCmd{makenote}! Currently the package cannot recognise
such a mistake, and it causes new shifts of the synchronisation point with
each \LaTeX{} run, so the process will never terminate. To synchronise two or
more note columns, you have to synchronise each of them with the main text.
The recommended command for this is described below.%
%
\begin{Example}
  Let's extend the example above by first adding a synchronisation point
  and then another paragraph with a comment:
\begin{lstcode}
    \syncwithnotecolumn[paragraphs]\bigskip
    \makenote[paragraphs]{%
      \protect\begin{contract}
        \protect\Clause{title={Humor of a Culture}}
        \setcounter{par}{0}%
        The humour of a joke can be determined by the
        cultural environment in which it is told.

       The humour of a joke can be determined by the
       cultural environment in which it acts.
      \protect\end{contract}
    }
    The cultural component of a joke is, in fact, not
    negligible. Although the political correctness of
    using the cultural environment can easily be
    disputed, nonetheless the accuracy of such comedy
    in the appropriate environment is striking. On
    the other hand, a supposed joke in the wrong
    cultural environment can also be a real danger
    for the joke teller.
\end{lstcode}
  In addition to the synchronisation point, a vertical skip has been added
  with \Macro{bigskip} to better distinguish each paragraph and the
  corresponding comments.

  Further\textnote{Attention}, this example illustrates another potential
  problem. Because the note columns uses boxes that are assembled and
  disassembled, counters\textnote{counter in note column} inside note columns
  can sometimes jitter. In the example, therefore, the first paragraph would
  be numbered 2 instead of 1. This, however, can easily be fixed by a direct
  reset of the corresponding counter.

  The example is almost complete. You just have to close the environments:
\begin{lstcode}
  \end{addmargin}
  \end{document}
\end{lstcode}
  In reality, of course, all the remaining section of the law should also be
  commented. But let us focus on the main purpose.
\end{Example}
But stop! What if, in this example, the \PName{paragraphs} would no longer fit
on the page? Would it be printed on the next page? We will answer this
question in the next section.%
\EndIndexGroup


\begin{Declaration}
  \Macro{syncwithnotecolumns}\OParameter{list of note column names}
\end{Declaration}
This command synchronises the main text with all note columns of the
comma-separated \PName{list of note column names}. The main text
will be synchronised with the note column whose mark is closest to the
end of the document. As a side effect, the note columns will be synchronised
with each other.

If the optional argument is omitted or empty (or begins with \Macro{relax}),
synchronisation will be done with all currently declared note columns.%
\EndIndexGroup


\section{Forced Output of Note Columns}
\seclabel{clearnotes}

In addition to the normal output of note columns described in the previous
section, you may sometimes need to output all collected notes that have
not yet been output. This is especially useful when large notes cause more and
more notes to be moved down to new pages. A good time to force the
output\textnote{force output} is, for example, the end of a chapter or the end
of the document.

\begin{Declaration}
  \Macro{clearnotecolumn}\OParameter{note column name}
\end{Declaration}
This command prints all notes\textnote{pending notes} of a particular note
column that have not yet been output by the end of the current page but which
were created on that or a previous page. Blank pages are generated as needed
to output these pending notes. During the output of the pending notes of this
note column, notes of other note columns may also be output, but only as
necessary to output the pending notes of the specified note column.

During the output of the pending notes, notes created in the previous \LaTeX{}
run on the pages that are now replaced by the inserted blank pages may be
output by mistake. This will be corrected automatically in one of the
subsequent \LaTeX{} runs. Such shifts are usually indicated by the message:
``\LaTeX{} Warning: Label(s) may have changed. Rerun to get cross-references
right.''

The note column whose pending notes are to be output is indicated by the
optional argument \PName{note column name}. If this argument is omitted, the
notes of the default note column \PValue{marginpar} will be output.

The attentive reader\textnote{forced output vs. synchronisation} will have
noticed that the forced output of a note column is not unlike synchronisation.
But if the forced output actually results in an output, it will be at the
start of a new page and not just below the previous output. Nonetheless, a
forced output usually results in fewer \LaTeX{} runs.%
\EndIndexGroup


\begin{Declaration}
  \Macro{clearnotecolumns}\OParameter{list of note column names}
\end{Declaration}
This command is similar to \DescRef{\LabelBase.cmd.clearnotecolumn}, but the
the optional argument here can be not only the name of one note column but a
comma-separated \PName{list of note column names}. The pending notes of all
these note columns are then ouput.

If you omit the optional argument or leave it empty, all pending notes of all
note columns will be output.%
\EndIndexGroup


\begin{Declaration}
  \OptionVName{autoclearnotecolumns}{simple switch}
\end{Declaration}
As a rule, pending notes will always be output if a document
implicitly\,---\,e.\,g. because of a \DescRef{maincls.cmd.chapter}
command\,---\,or explicitly executes \DescRef{maincls.cmd.clearpage}. This is
also the case at the end of the document within
\Macro{end}\PParameter{document}. The \Option{autoclearnotecolumns} option can
be used to control whether \DescRef{\LabelBase.cmd.clearnotecolumns} should be
executed automatically without arguments when running
\DescRef{maincls.cmd.clearpage}.

Since this is generally the desired behaviour, the option is active by
default. But you can change this with the appropriate value for a simple
switch (see \autoref{tab:truefalseswitch}, \autopageref{tab:truefalseswitch})
at any time.

Note\textnote{Attention!} that disabling the automatic output of pending notes
can result in lost notes at the end of the document. So in this case you
should insert \DescRef{\LabelBase.cmd.clearnotecolumns} before
\Macro{end}\PParameter{document} fore safety's sake.

The question from the end of the last section should now be answered. If the
paragraph is to be completely output, it had to be wrapped to the next page.
This is, of course, the default setting. However, since this would happen
after the end of the \DescRef{maincls.env.addmargin} environment, the forced
output could still overlap with subsequent text. So in the example it would
make sense to add another synchronisation point after the
\DescRef{maincls.env.addmargin} environment.

The result of the example is shown in
\autoref{fig:scrlayer-notecolumn.example}.\iffree{}{ Only the colour has been
changed from blue to grey.}

\begin{figure}
  \KOMAoptions{captions=bottombeside}%
  \setcapindent{0pt}%
  \begin{captionbeside}[{A sample page for the example in
      \autoref{cha:scrlayer-notecolumn}}]{A sample page for the example in
      this chapter\label{fig:scrlayer-notecolumn.example}}[l]
  \frame{\includegraphics[width=.5\textwidth,keepaspectratio]%
    {scrlayer-notecolumn-example}}
  \end{captionbeside}
\end{figure}
\EndIndexGroup
%
\EndIndexGroup

%%% Local Variables:
%%% mode: latex
%%% mode: flyspell
%%% ispell-local-dictionary: "en_GB"
%%% TeX-master: "../guide"
%%% End: 
