% xmlparser.tex
%%%%%%%%%%%%%%%%%%
% Petr Olsak  2016

% After \input xmlparser you can do:

% \xmlprep {domument.xml} {document.out}
% You can define all tags used in the document.xml 
% in the form \def\XMLtag#1#2{...}.  Then you can process:
% \input document.out

% The macro \xmlprep {input.xml} {output-file} converts XML document to a
% TeX-friendly format. You can define used macros and do \input output-file. 
% More information is at the end of this document

\newwrite\xmloutfile
\def\xmlprep#1#2 % normal %
\def\xprint#1{\immediate\write\xmloutfile{\xindent#1}}
\def\xindent{}

\def\xtag#1{\ifx#1!\expandafter\xtagH \else\fihere\xtagA#1\fi}
\def\xtagA#1#2>{\ifx#1?\xtagE#2>\else\ifx#1/\xtagG#2>\else\xtagB#1#2>/>\end\fi\fi}
\def\xtagB#1/>#2\end{\ifx>#2>\let\tmp=n\xtagC#1 \end\else \let\tmp=/\xtagC#1> \end\fi}
\def\xtagC#1 #2\end{\def\currargs{}\ifx>#2>\xtagD#1\else \xtagF#2\xtagD#1>\fi}
\def\xtagD#1>{\bgroup\def\currtag{#1}%
   \ifx\tmp/\xprint{\string\XML#1\space{\currargs}{}}\egroup\else
   \xprint{\string\XML#1\space{\currargs}\iftrue\string{\else}\fi\npercent}%
   \edef\xindent{\xindent\space\space}\fi
}
\def\xtagE#1?>{\xprint{\string\META\space{#1}}}
\def\xtagF#1>{\def\currargs{#1}}
\def\xtagG#1>{\def\tmp{#1}\ifx\tmp\currtag\else
   \message{WARNING: <\currtag>...</#1> doesn't match}\fi
   \egroup\xprint{\iffalse{\else\string}\fi\npercent}%
}
\def\xtagH#1{\ifx#1-\expandafter\xtagI \else \fihere\xtagJ#1\fi}
\def\xtagI#1-->{} % comment in the format <!-- ... -->
\def\xtagJ#1 #2>{\xprint{\string\SPEC#1\space{#2}}}
\def\fihere#1\fi{\fi#1}

\def\xarg#1{\xargA#1 ==}
\def\xargA#1#2={\def\xargN{#1#2}\ifx#1=\else\expandafter\xargB\fi}
\def\xargB#1{\ifx#1"\expandafter\xargC\else\fihere\xargE#1\fi}
\def\xargC#1"{\xargD{#1}}
\def\xargD#1{\expandafter\def\csname ARG\xargN\endcsname{#1}\xargA}
\def\xargE#1{\ifx#1'\expandafter\xargF\else\fihere\xargG#1\fi}
\def\xargF#1'{\xargD{#1}}
\def\xargG#1 {\xargD{#1}}

\def\META#1{}  \def\SPECDOCTYPE#1{}

\def\entity#1;{\csname ent:#1\endcsname}
\def\declentity#1#2{\expandafter\def\csname ent:#1\endcsname{#2}}

\endinput

--------------------------------------------------------------------------

Documentation
=============

Introduction example. Suppose the test.xml file:

--------------------

<?xml version="1.0" encoding="utf8"?>
<pricelist>
  <!-- This is price list of a virtual firm -->
  <name>Computer components</name>
  <validity from="1.1.2000" to="31.3.2000"/>
  <firm>
    <name>První hardwarová, s.r.o.</name>
    <address>
      <street>Průmyslová 12</street>
      <city>Praha 10</city>
      <postalcode>100 000</postalcode>
      <email>info@prhv.cz</email>
    </address>
  </firm> 
  <offer>
    <product category="polohovací zařízení" code="pxbd-21">
      <name>Hyperoptická <em>digitální</em> myš</name>
      <price currency="CZK">368.30</price>
    </product>
    <product category="pevné disky" code="sbhd-99">
      <name>Soft-slow disc &lt; 19,3 GB</name>
      <price currency="CZK">8500</price>
    </product>
    <product category="polohovací zařízení" code="pxbd-13">
      <name>Special touchpad</name>
      <price currency="CZK">5635.20</price>
    </product>
  </offer>
</pricelist>

--------------------

When you process it by \xmlprep {test.xml} {test.out} you get:

--------------------

\META {xml version="1.0" encoding="utf8"}
\XMLpricelist {}{%
  \XMLname {}{%
    Computer components%
  }%
  \XMLvalidity {from="1.1.2000" to="31.3.2000"}{}
  \XMLfirm {}{%
    \XMLname {}{%
      První hardwarová, s.r.o.%
    }%
    \XMLadddress {}{%
      \XMLstreet {}{%
        Průmyslová 12%
      }%
      \XMLcity {}{%
        Praha 10%
      }%
      \XMLpostalcode {}{%
        100 000%
      }%
      \XMLemail {}{%
        info@prhv.cz%
      }%
    }%
  }%
  \XMLoffer {}{%
    \XMLproduct{category="polohovací zařízení" code="pxbd-21"}{%
      \XMLname {}{%
        Hyperoptická %
        \XMLem {}{%
          digitální%
        }%
        myš%
      }%
      \XMLprice {currency="CZK"}{%
        368.30%
      }%
    }%
    \XMLproduct {category="pevné disky" code="sbhd-99"}{%
      \XMLname {}{%
        Soft-slow disc &lt; 19,3 GB%
      }%
      \XMLprice {currency="CZK"}{%
        8500%
      }%
    }%
    \XMLproduct {category="polohovací zařízení" code="pxbd-13"}{%
      \XMLname {}{%
        Special touchpad%
      }%
      \XMLprice {currency="CZK"}{%
        5635.20%
      }%
    }%
  }%
}%

--------------------------

This format is more comfortable for further TeX processing.
You can define appropriate macros, for example:

--------------------------

\newtoks\street \newtoks\city \newtoks\postalcode \newtoks\email

\def\XMLpricelist#1{} % only process second argument ...
\def\XMLname#1#2{{\bf#2}\medskip}
\def\XMLvalidity#1#2{}
\def\XMLfirm#1#2{\bgroup
   \def\XMLname##1##2{{\it##2}\par}%
   \def\XMLaddress##1##2{##2\printaddress}%
   \def\XMLstreet##1##2{\street{##2}}%
   \def\XMLcity##1##2{\city{##2}}%
   \def\XMLpostalcode##1##2{\postalcode{##2}}%
   \def\XMLemail##1##2{\email{##2}}%
   \def\printaddress{ulice: \the\street, 
      mesto: \the\postalcode\space\the\city, email: {\tt\the\email}} 
   Dodavatel: #2\par
   \egroup
}
\def\XMLoffer#1{}
\def\XMLproduct#1#2{\bgroup
   \xarg{#1}%
   \def\XMLname##1##2{\def\name{##2}}%
   \def\XMLprice##1##2{\xarg{currency=?}\xarg{##1}\def\price{##2}}
   #2%
   \centerline{\name\space(\ARGcode)\dotfill\price\space\ARGcurrency}
   \egroup
}
\def\XMLem#1#2{{\it#2} \ignorespaces}

\catcode`&=13 \let&=\entity
\declentity{lt}{$<$}

\input test.out

----------------------------

After \input test.out, you get the desired document.


Features of the \xmlprep conversion
===================================

The <tag arguments>text</tag> is converted to:

   \XMLtag {arguments}{%
      text%
   }%

The <tag arguments/> or <tag/> are converted to

   \XMLtag {arguments}{}  or  \XMLtag {}{}

The <!-- text --> is ignored.

The <?text?> is converted to:

   \META {text}

The <!TEXT text> is converted to:

   \SPECTEXT {text}

The closings </tags> are checked to the opening <tag>. 
The nested tags are indented in the output.


Scanning of the arguments
=========================

The <tag arguments> are converted to \XMLtag{arguments}{%, so the arguments
are saved as first parameter of the \XMLtag macro. Arguments are typically
in the form

   argA="valueA" argB="valueB" argC="valueC"

You can define

   \def\XMLtag#1#2{\bgroup \xarg{#1}...process #2\egroup}

The \xarg{arguments} scans the arguments given in the parameter. The result of
\xarg{"valueA" argB="valueB" argC="valueC"} is equivalent to
\def\ARGargA{valueA}\def\ARGargB{valueB}\def\ARGargC{valueC}, so you can use
these macros in further processing.

There are alternatives of the format of arguments:

   argA='valueA' or argA=valueA (separated by space or end of arguments).

The \xarg macro is able to treat these alternatives properly.

Recommendation: If you assume default values of arguments then do something
similar to this:

   \xarg{argA="defaultA" argB="defaultB"}\xarg{#1}.


XML entities
============

The XML text includes sometimes the entity in the form &name; If it is true
in your XML document, then you can set & as active with the meaning \entity 
and declare the used entities by \declentity{name}{what to do}. If the document 
includes the entities &l; &gt; &amp; (for example) then you can do:

   \catcode`&=13 \let&=\entity
   \declentity{lt}{$<$}
   \declentity{gt}{$>$}
   \declentity{amp}{\&}


Various approaches for \XMLtag definitions
==========================================

Classical approach is

   \def\XMLtag#1#2{\bgroup process arguments #1, process body #2\egroup}

You can define \XMLtag with only one parameter. Then the second parameter is
normally processed in the group. This is usable if you need to keep the
possibility of catcode changing in the document. If you need to process
arguments in the same group as body then do something like this:

   \def\XMLtag#1{\bgroup \xarg{#1}\let\next=}

If you read body in #2 then you can decide what to do before and what to do
after:

   \def\XMLtag#1#2{what to do before #2 what to do after}

If the body of <tag> includes data declared in <tagA>, <tabB>, then you can
save the data first and then print the result in the part "what to do after". 
Example:

   <tag> <tagA>somethingA</tagA> <tagB>somethingB</tagB> </tag>

   \newtoks\dataA \newtoks\dataB
   \def\XMLtag#1#2{\bgroup
      \def\XMLtagA##1{\dataA=}\def\XMLtagB##1{\dataB=}%
      #2% process the body, dataA and dataB are set.
      print \the\dataB and \the\dataA.
      \egroup}

The meaning of the tag can depends on the outer tag used. See the "name" tag
in the introduction example. Then you must to define various meaning of such
tag inside another \XMLtag macro. 

   <tagA> <name>My name</name> </tagA> <tagB> <name>City name</name> </tagB>

   \def\XMLtagA#1#2{\bgroup
       \def\XMLname##1##2{...}%
       ...
       \egroup
   }
   \def\XMLtagB#1#2{\bgroup
       \def\XMLname##1##2{...something different}%
       ...
       \egroup
   }

-------------------------------------------------------

